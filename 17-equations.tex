\documentclass[12pt]{article}

\DeclareMathSizes{12}{30}{16}{12}

\usepackage{chngcntr}
\usepackage{amsmath}
\usepackage[T1]{fontenc}
\usepackage{mathtools}  % loads »amsmath«
\usepackage{physics}

\usepackage[colorlinks=true, urlcolor=blue, linkcolor=red]{hyperref}
%% https://www.baeldung.com/cs/latex-hyperref-url-hyperlinks

\counterwithin*{equation}{section}
\counterwithin*{equation}{subsection}

%%  Useful LaTeX Links
%%
%% https://tex.stackexchange.com/questions/5148/how-can-i-change-the-font-size-in-math-equations
%% https://en.wikibooks.org/wiki/LaTeX/Mathematics
%% https://www.overleaf.com/learn/latex/Mathematical_expressions
%% https://tug.ctan.org/info/short-math-guide/short-math-guide.pdf
%% https://quickref.me/latex
%% 


\begin{document}

\section*{Introduction}

%% * https://www.businessinsider.com/17-equations-that-changed-the-world-2014-3?op=1

Also see:

\begin{itemize}
	\item \href{https://www.businessinsider.com/17-equations-that-changed-the-world-2014-3}{businessinsider.com} article about the book
   	\item A summary of the \href{https://i.insider.com/53289f55eab8ea3d524b5f3a}{equations}
	\item \url{https://en.wikipedia.org/wiki/In_Pursuit_of_the_Unknown}
	\item \url{https://en.wikipedia.org/wiki/Ian_Stewart_(mathematician)}
	\item \url{https://en.wikipedia.org/wiki/Euler_Book_Prize}
	\item \url{https://github.com/Jonathanseng/17-Equations-that-change-the-world}
	\item \url{https://github.com/peterlharding/17-equations-that-changed-the-world}
\end{itemize}

\newpage
	
\section{The Square on the Hypotenuse}
\subsection{Pythagoras' Law}

``The square of the hypotenuse is equal to the sum of the squares of the other two sides.''

\vspace{2em}

\begin{equation*} a^2 = b^2 + c^2  \end{equation*}

\vspace{2em}

\url{https://en.wikipedia.org/wiki/Pythagoras}
	
\newpage


\section{Shortening the Proceedings}
	
Logarithms - John Napier, 1610

\vspace{2em}

\begin{equation*} \log xy = \log x + \log y \end{equation*}

\vspace{2em}

\url{https://en.wikipedia.org/wiki/John_Napier}

\newpage


\section{Ghosts of Departed Quantities}
	
Calculus - Issac Newton, 1668

\vspace{2em}

\begin{equation*} \frac{df}{dt} =\lim_{h \to 0}  \frac{f(t + h) - f(t)}{h} \end{equation*}

\vspace{2em}

%% https://www.overleaf.com/learn/latex/Integrals%2C_sums_and_limits

\url{https://en.wikipedia.org/wiki/Isaac_Newton}

\newpage


\section{The System of the World}

Law of Gravity - Issac Newton, 1687

\vspace{2em}

\begin{equation*}F = G \frac{m_1 m_2}{r^2}\end{equation*}

%% $^{a}\Big/_{b}$
%% https://tex.stackexchange.com/questions/4624/a-symbol-for-the-quotient-of-two-objects

\vspace{2em}

\url{https://en.wikipedia.org/wiki/Isaac_Newton}

\newpage


\section{Portent of the Ideal World}

The Square Root of minus One - Leonhard Euler, 1750

\begin{equation*} i^2 = -1 \end{equation*}

\vspace{2em}

\url{https://en.wikipedia.org/wiki/Leonhard_Euler}

\newpage
	

\section{Much Ado about Knotting}
	
Euyler's Formula for Polyhedra - Leonhard Euler, 1751

\vspace{2em}

\begin{equation*} V -E + F = 2 \end{equation*}

\vspace{2em}

\url{https://en.wikipedia.org/wiki/Leonhard_Euler}

\newpage


\section{Patterns of Chance}
	
Normal Distribution - Carl Friedrich Gauss, 1810

\vspace{2em}

\begin{equation*}\Phi{(x)} = \frac{1}{\sqrt{2 \pi \sigma}} \ e^\frac{(x - \mu)^2}{2 \sigma^2}\end{equation*}

\vspace{2em}

\url{https://en.wikipedia.org/wiki/Carl_Friedrich_Gauss}

\newpage


\section{Good Vibrations}
	
Wave Equation - Jean le Rond d'Alembert. 1746

\vspace{2em}

	\begin{equation*}\frac{\partial^2 u}{\partial t^2} = c^2 \  \frac{\partial^2 u}{\partial x^2} \end{equation*}

%% https://www.overleaf.com/learn/latex/Mathematical_expressions
%% https://www.math-linux.com/latex/faq/latex-faq/article/latex-derivatives-limits-sums-products-and-integrals
%% https://tex.stackexchange.com/questions/162273/second-derivative
%% https://www.physicsread.com/latex-derivatives/
%% https://www.maths.tcd.ie/~dwilkins/LaTeXPrimer/Calculus.html

\vspace{2em}

\url{https://en.wikipedia.org/wiki/Jean_le_Rond_d%27Alembert}

\newpage


\section{Ripples and Blips}
	
Fourier Transforms - Joseph Fourier, 1822

\vspace{2em}

\begin{equation*} \hat{f}\relax(\xi) = \int_{-\infty}^{\infty} f\relax(x) \  e^{-2\pi i x \xi} dx \end{equation*}

\vspace{2em}

\url{https://en.wikipedia.org/wiki/Joseph_Fourier}
	
\newpage


\section{The Ascent of Humanity}

Navier-Stokes Equation - Claude-Louis Navier, George Stokes, 1845

%% https://kbwiki.ercoftac.org/w/index.php/Latex_Equations_Cribsheet
%% https://www.overleaf.com/learn/latex/Brackets_and_Parentheses

\vspace{2em}

\begin{equation*}   \rho \left( \frac{\partial{\mathbf{v}}}{\partial{t}} + \mathbf{v} \cdot \grad{\mathbf{v}} \right) = - \grad{p} + \div \mathbf{T} +  \mathbf{f}  \end{equation*}

or using Einstein notation:

\begin{equation*}   \pdv{(\rho \vec{u})}{t}+\vec{\nabla}\cdot\rho\vec{u}\otimes\vec{u} = -\vec{\nabla p}+\vec{\nabla}\cdot\Bar{\Bar{\tau}}+\rho\vec{f}  \end{equation*}

\begin{itemize}
	\item \url{https://en.wikipedia.org/wiki/Claude-Louis_Navier}
	\item \url{https://en.wikipedia.org/wiki/Sir_George_Stokes,_1st_Baronet}
	\item \url{https://en.wikipedia.org/wiki/Navier%E2%80%93Stokes_equations}
\end{itemize}

\newpage


\section{Waves in the Ether}
	
Maxwell's Equations - James Clerk Maxwell, 1865

\vspace{2em}

\begin{equation*}
	\begin{aligned}
		\frac{\partial\mathcal{D}}{\partial t} \quad & = \quad \nabla\times\mathcal{H},   & \quad \text{(Faraday's Law)} \\[5pt]
		\frac{\partial\mathcal{B}}{\partial t} \quad & = \quad -\nabla\times\mathcal{E},  & \quad \text{(Ampère's Law)}   \\[5pt]
		\nabla\cdot\mathcal{B}                 \quad & = \quad 0,                         & \quad \text{(Gauss's Law)}   \\[5pt]
		\nabla\cdot\mathcal{D}                 \quad & = \quad 0.                         & \quad \text{(Colomb's Law)}
	\end{aligned}
\end{equation*}

or

\begin{equation*}
	\begin{aligned}
\div \mathbf{E} = 0 \\
\div \mathbf{H} = 0 \\
\curl{\mathbf{E}} = - \frac{1}{c} \frac{\partial \mathbf{H}}{\partial t} \\
\curl{\mathbf{H}} = - \frac{1}{c} \frac{\partial \mathbf{E}}{\partial t} \\
	\end{aligned}
\end{equation*}

where $ \mathbf{E}$ is the Electric Field and $\mathbf{H}$ is the magnetic field.

\vspace{2em}

\url{https://en.wikipedia.org/wiki/James_Clerk_Maxwell}

\newpage


\section{Law and Disorder}

Second Law of Thermodynamics - Ludwig Boltzmann, 1874

\vspace{2em}

\begin{equation*}  d \mathbf{S} <= 0  \end{equation*}

\vspace{2em}

\url{https://en.wikipedia.org/wiki/Ludwig_Boltzmann}

\newpage


\section{One Thing is Absolute}

Relativity - Albert Einsten, 1905

\vspace{2em}

\begin{equation*}E=mc^2 \end{equation*}

\vspace{2em}

\url{https://en.wikipedia.org/wiki/Albert_Einstein}

\newpage


\section{Quantum Weirdness}

Schrodinger's Equation - Edwin  Schrodinger, 1927

\vspace{2em}

\begin{equation*}   i \hbar \  \frac{\partial}{\partial{t}} \  \Phi = \hat{H} \Phi \end{equation*}

\vspace{2em}

\url{https://en.wikipedia.org/wiki/Erwin_Schr%C3%B6dinger}

\newpage


\section{Codes, Communications and Computers}

Information Theory - Claude Shannon, 1949

\vspace{2em}

\begin{equation*} H = \sum_{x} p(x) log_{2}\ p(x) \end{equation*}

\vspace{2em}

\url{https://en.wikipedia.org/wiki/Claude_Shannon}

\newpage


\section{The Imbalance of Nature}

Chaos Throry Theory - Robert May, 1975

\vspace{2em}

\begin{equation*} x_{t+1}  = kx_t ( 1 - x_t) \end{equation*}

\vspace{2em}

\url{}

\newpage



\section{The Midas Formula}
				
The Black-Scoles Equation - F. Black \& M. Scholes - 1990

\vspace{2em}

\begin{equation*}  \frac{1}{2} \sigma^2 S^2 \frac{\partial_2V}{\partial{S_2}} + r S \frac{\partial{V}}{\partial{S}} +  \frac{\partial{V}}{\partial{t}} - r V = 0 \end{equation*}

\vspace{2em}

\url{https://en.wikipedia.org/wiki/Black%E2%80%93Scholes_model}

\newpage 
	


	
\section*{Extras}

%% https://www.overleaf.com/learn/latex/Fractions_and_Binomials


The binomial coefficient, \(\binom{n}{k}\), is defined by the expression:

\[
\binom{n}{k} = \frac{n!}{k!(n-k)!}
\]
	
\end{document}